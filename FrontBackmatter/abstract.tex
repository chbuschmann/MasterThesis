\thispagestyle{empty}
\chapter*{Abstract}
The music of science fiction. Why did it become to sound the way it does and how does the musical mechanisms work? How are space, the extraordinary, the fantastic and the outer-worldly depicted in western science fiction filmography? This thesis takes a historical and analytical perspective as it looks closer on the musical constructions we find in the music of the movies of \textit{Star Trek}. Using conventional music analytical tools combined with \acf{nRT} and \textit{transformational theory} this thesis hopes to shed light on the inner workings in the music of \textit{Jerry Goldsmith}, \textit{James Horner} and \textit{Michael Giacchino} in the iconic films: ``Star Trek: The Motion Picture'', ``Star Trek: The Wrath of Khan'', ``Star Trek: First Contact'', ``Star Trek: Nemesis'', ``Star Trek'' and ``Star Trek: Into Darkness''.