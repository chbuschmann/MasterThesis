% !TEX root = ../Masterthesis.tex

\thispagestyle{empty}
\begin{fullwidth}
\chapter*{Orthography and Examples}\label{ch:orthography}

Due to the fact that procuring the original scores was beyond my reach, I have chosen to transcribe all of the musical examples provided in this thesis. I have done my utmost to provide as accurate a rendition as possible; however, some things may have been omitted for the sake of clarity or because they simply were not heard. There are a few challenges worth nothing: Octave doublings and woodwinds are especially hard to hear through the thick orchestration. Time signatures and enharmonisism will, at all times, reflect the current mood of the author and might differ from the original scores.

\begin{table*}
	\begin{tabularx}{\textwidth}{ll}
	\textbf{Symbol} & \textbf{Translation} \\
\toprule
	C & C major triad \\
	Cm & C minor triad \\
	\chord{C}{maj7} & C with added diatonic 7 \\
	\chord{C}{7} & C with added flat 7 \\
	C\(^{7(\tinysharpx 5)}\) & C with added flat 7 and sharp 5 \\
	\bigchord{C}{7}{\tinysharpx 9}{\tinysharpx 5}& C with added flat 7 and sharp 9 and sharp 5 \\
	\chord{C}{11} & C with added flat 7, 9, 11 \\
	\chord{C}{13} & C with added flat 7, 9, 11 and 13 \\
	C/B\flatx & C with B\flatx{} in bass \\
	\(\frac{C}{D} \) & C major over D major \\
	C\(^{7}_{(a)b/c/d}\) & (root), first, second, third inversion \\
	\textit{pc} [0,2,4,T] & Absolute pitch: C, D, E, B\flatx \\
	\(\hat{1}\) & Relative to chord \\
	\fbox{C}: & The key of C \\
	\(\int\) & Substitute symbol: \((\int){m}^{7(\tinyflatx{5})}\) \\
\bottomrule
\end{tabularx}  
    \label{tb:orthography}
\end{table*}

\end{fullwidth}