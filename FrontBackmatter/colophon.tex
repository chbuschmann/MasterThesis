% !TEX root = ../../../latex i musikkvitenskap/the musical conventions of star trek/Masterthesis.tex

\chapter{Colophon}
\thispagestyle{empty}

\begin{fullwidth}

\begin{lstlisting}
\chapter{Colophon} 
\thispagestyle{empty}

\noindent\newthought{This book} was typeset with \LaTeX{} on \url{http://www.overleaf.com}. The design is based on the Tufte-\LaTeX{} book class available at \url{https://github.com/Tufte-LaTeX/tufte-latex}. The main font is Palatino. In addition Helvetica and Bera Mono are used. I use the \texttt{lilyglyps} package for musical symbols and the \texttt{xpiano} package for keyboard illustrations. The diagrams was made using Apple's Keynote and the music was engraved with Sibelius 7.5.

\end{lstlisting}

\noindent\newthought{This book} was typeset with \LaTeX{} on \url{http://www.overleaf.com}. The design is based on the Tufte-\LaTeX{} book class available at \url{https://github.com/Tufte-LaTeX/tufte-latex}. The main font is Palatino. In addition Helvetica and Bera Mono are used. I use the \texttt{lilyglyps} package for musical symbols and the \texttt{xpiano} package for keyboard illustrations. The diagrams was made using Apple's Keynote and the music was engraved with Sibelius 7.5.

\end{fullwidth}


% Reviewed
