% !TEX root = ../../Masterthesis.tex
\chapter{Conclusion}\label{ch:conclusion}

\newthought{''Modern Film Scores - The McDonald's of Music?''} was the title I was going to use for my conclusion, with the subheading: ''Or the New American Minimalism?''. With this statement I wished to convey a rather unpleasant suspicion the film music industry are subject to more and more ''fast food music''. Is it really so? In some ways I believe proof of this trend have been partially confirmed during the course of this study. There have indeed been a change on how film and music work together and it would seem that it has to do with how the movie industry as whole has changed: lesser time to produce great movies. With the ''gonzo'' effect I have mentioned before, we see a clear trend of repeating and simplifying musical content and using texture to fill the ''quota'', so to speak. There would also seem that there is a trend of creating harmonic progressions and content based on the circle of fifths, progressions usually confined to the more popular segment of musical industry. This leads me to the question: On of film musics greatest strengths is its liberty to go \textit{beyond} regular sonorities \textit{because} it is independent of the audience; it is fully and wholly under the power of the screen. But is it so that the current belief is that the more tonally complex, the narrower impact you have on your audience? From my study alone it is impossible to tell. What I \textit{can} tell is that there is less music and more ''musical noise'' coming from the newer movies. Maybe this is part of the answer of my subtitle; It would seem the musical syntax in current science fiction is following a tendency of less variation and more generalization. But in the end, does it mean anything? Perhaps, if its reason is to submit to the people.

This study has but uncovered an extremely small percentage of the total sum and there is no need to raise the alarms; this is just a early observation and ultimately I am talking about possible trending tendencies. Even so, there is no denying the fact that we do indeed see a trend of declining harmonic and melodic complexity, but there are lots of examples of movie and TV scores that are as diverse, if not more so, as the ''old'' scores of the industry giants, Korngold, Steiner and Williams, to name but a few. And even the new Star Trek scores: they have parts that are absolutely beautiful. Music are really about patterns and redundancy; certain patterns have stuck with us for a long time when we hear new patterns we are quite skeptical at first. Therefore it is fair to present a more optimistic view on the matter: Perhaps we are living the paradigm shift we have seen between Baroque and Rococo, Bop and Cool, Serialism and Minimalism? Only time will tell where this is heading.

As a final note I wish to applaud the strengths of \textit{neo-Riemannian Theory}. It has proven a most capable tool for analyzing patterns, and I hope that others will use it as well to explore its boundaries on and beyond. 

\vspace{5cm}
\center
Live long, and prosper

