% !TEX root = ../../Masterthesis.tex

\chapter{The Musical Conventions of Star Trek}\label{ch:the musical conventions}

\newthought{I have only} scratched the surface regarding the actual musical syntax of Star Trek. I have, however, observed a few things worth noting. 

In many ways music can be treated as semantics; certain musical devices has become synonymous with a certain image. Take John Williams ``Jaws'' theme, or the use of the lydian mode to portrait something ``magical''. With that in mind, what does the music of Star Trek tells us? It tells us that harmonic progressions are very much a product of thirds. Scales like the lydian, mixolydian \flatx6 and the octatonic scale rule supreme in underscores. This in turn tells us that the composers uses more predicable tonal constructs when dealing with music that is in front of the narrative. It also tells us about a gradual harmonic reduction. \textbf{ST:TMP} and \textbf{The Wrath of Kahn} is a harmonic bonanza compared to that found in \textbf{Star Trek (2009)} and \textbf{Into Darkness}. It would be easy to draw a line that starts on top of the harmonic complexity scale at 1979, almost modernistic in places, and are quite a bit down the scale in 2013; a gradual decline. 

It is logical to assume that \textit{why} the music overall becomes simpler, and perhaps more ``efficient'' constructs over time is because movies has changed to adapt to new movie audiences that has grown accustomed to \textit{''gonzo''} entertainment. This might be a case of evolutionistic harmonic redundancy - the audience instinctively knows what to expect after a given amount of musical time therefore the composers skips a bit on the harmonic justification. It is comparable with the jump in conclusions we see from Bach and the baroque music, where every chord had to have a function and was thoroughly justified before executed, and Wagner, who \textit{assumed} the functional logic and jumped straight to the conclusion. Also, this might be as ``simple'' as drastic cuts in production time leaving the composer bound to produce \textit{something} in a short amount of time, however, I believe it to be a combination of all of the above.

Regarding the inner layers of orchestration we see that the ``John Williams'' era is transforming into something else; A new direction. We no longer see the exuberant sweeping strings and multiple counterpoints spread throughout the orchestra found in Goldsmith and Horner's work from the 80's and 90's. Instead we see a immensely thick orchestration with lots of instruments doubling each other stating the music mono thematically. 

If we look briefly at the internal musical structures, they has flattened dramatically. The main title of \textbf{ST:TMP} uses a variant of the rondo form to build and develop themes. With the main title of \textbf{Nemesis}, Goldsmith uses but one theme that he barely develops before hinting at another theme in the end. The main title of \textbf{Star Trek (2009)} differs in that it refuses to reuse the old themes further differentiating it self from the franchise. Only hints and homages to progressions are to be found. The main title features two themes, the first one stated twice and the second one, a one bar loop, repeats ad infinitum. 

The size of the orchestra has of course evolved but in essence remains much the same; gigantic. \textbf{Wrath of Kahn} was recorded with a 91-piece orchestra and \textbf{Star Trek (2009)} was recorded with a 107-piece orchestra and 40-person choir. The same goes for the use of synths and other sound effects as part of the score. Goldsmith used several synthesizers and organs to expand his sonority palette, Giacchino uses different synths to add texture as well. The use of non-orchestral textures is evolving along side technology and are providing composers with new tools to express emotions. 
